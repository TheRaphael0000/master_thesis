\section{Evaluation \label{sec:evaluation}}

This section explains the divers evaluation metrics used and experiments realized on the datasets to evaluate the proposed methods (ref. Section~\ref{sec:methods}).
Each experiment of this section is described by its purpose in the study, the methodology used to conduct it, the results obtained in the form of: graphic, table or statistics, an analysis of the results and sometimes a few additional conclusion to the experiment.
The section is seperated into four parts: one concerning the creation and the evaluation of rank lists acording to the proposed methods, an analysis on the errors encountered, and another one dedicated to the clustering task.

Experiments realized in this study were written in the Python programming language (version 3.8.8)~\cite{python}, using the following libraries:
\begin{itemize}
  \item Python Standard Library 3.8.8, for the data types, functional programming, file and direcotry access, data compression, text processing~\cite{python_standard_library}
  \item Matplotlib 3.3.4, for the plot generations~\cite{matplotlib}
  \item Numpy 1.20.2, for scientific computations~\cite{numpy}
  \item SciPy 1.4.1, for scientific computations~\cite{scipy}
  \item Scikit-learn 0.24.1, for machine learning algorithms implementations~\cite{sklearn}
  \item bcubed 1.5, to compute the BCubed family metrics~\cite{bcubed_gh}
  \item adjustText 0.7.3, for automatics text adjustments in plots~\cite{adjustText}
  \item tqdm 4.60.0, to have progress bars for heavy computations~\cite{tqdm}
\end{itemize}
The methods exposed in Section~\ref{sec:methods} are written in seperate Python files (extension .py).
Instead the experiments presented in this section, are contained in a special file format called IPython Notebook (extension .ipynb).
A Jupyter environement is required to run these files.
For instance JupyterLab allows markdown annotation, the possibility to run only portions of codes and even new code in the same Python kernel this is in particular useful to avoid having to run heavy computation multiple times during the implementation and optimizations the methods.

\subfile{1_rank_lists}
\subfile{2_fusion}
\subfile{3_clustering}
\subfile{4_results_analysis}
