\subsection{Errors analysis}

This section aim to understand the errors of the system, provide potential cues to solve them.

\subsubsection{Frequent errors}
\label{sec:frequent_errors}

This section try to understand the errors in the system, in this case the false links (document pairs with different authors) highly ranked on different rank lists.
The rank list quality is highly based on the feature vector created using the $n$-MFW, having a good understanding of this vector give good indications of the strength of the system.

To find recurrent errors in the system, the retained methods in Section~\ref{sec:mfw_token_lemma} on the relative frequency of the 750-MFW using the whole St-Jean dataset.
This generates the 9 rank lists, presented in Table~\ref{tab:9rl_results_st_jean_A_B} in annex.
For the sake of this experiment, every true links are discarded in the rank lists.
Links often apearing in the top 30 false links are considered as frequent errors.
The goal is to compare the two feature vectors of the frequent errors.
In this case, the visualization is based on the 750-MFW tokens representation.
Figures~\ref{fig:mfw_vector_error} show two pairs (document 57 from Sand / document 141 from Stael and document 57 from Sand / document 183 from Stael) that appear 6 time out of the 9 rank lists in the top 30 false links.
To be able to understand more easily this vector, the values have been sorted by relative frequencies and using the logarithmic scale.
When a large proportion of the vectors overlap it indicates a high similarity between the MFW vectors.

\begin{figure}
  \centering
  \caption{Example of 750-MFW relative frequency vectors for the two documents in a reccurant (6 rank lists out of 9) false link in the top 30 false links}
  \label{fig:mfw_vector_error}

  \subcaption{First example}
  \label{fig:mfw_vector_error_0}
  \includegraphics[width=\linewidth]{img/mfw_vector_error_0.png}

  \subcaption{Second example}
  \label{fig:mfw_vector_error_1}
  \includegraphics[width=\linewidth]{img/mfw_vector_error_1.png}
\end{figure}

Both document style are close when their feature vector are closely related.
Visually when most the surface overlap, the distance function will give a low value, and rank this link high in the list.
When two vectors are relatively close together determinating that two texts are from a different author can not clearly be established using only one type of representation, no matter the distance metric applied.
The frequent errors vector pairs presented in Figure~\ref{fig:mfw_vector_error} can be visually compared actual true links and actual false links, to have a better understanding of this problematic.
For example the most similar true link (ranked 1 using Manhattan distance) in Figure~\ref{fig:mfw_vector_first_rl} (document 157 from Stael / document 183 from Steal) or the HPrec-th (last continous correct pair from the top of the list) in Figure~\ref{fig:mfw_vector_first_last_rl} (document 22 from Gautier / document 53 from Gautier), both of these links show a large proportion of overlapping surface, like for the frequent errors vectors.
A counter example would be the least similar false link (ranked last using Manhattan distance) which represent a negatively correlated document pair, Figure~\ref{fig:mfw_vector_last_rl} showcase this link (document 183 from Stael / document 194 from Regnier).
As expected, most of this figure surface is non-overlapping.

\begin{figure}
  \centering
  \caption{750-MFW relative frequency for the two documents ranked $X$ in the rank list using the token representation on St-Jean}

  \subcaption{$X$ = First}
  \label{fig:mfw_vector_first_rl}
  \includegraphics[width=\linewidth]{img/mfw_vector_first_rl.png}

  \subcaption{$X$ = HPrec-th}
  \label{fig:mfw_vector_first_last_rl}
  \includegraphics[width=\linewidth]{img/mfw_vector_first_last_rl.png}

  \subcaption{$X$ = Last}
  \label{fig:mfw_vector_last_rl}
  \includegraphics[width=\linewidth]{img/mfw_vector_last_rl.png}
\end{figure}

\subsubsection{Publication date differences analysis}

When dealing with false links ranked high in the rank list, as the previous experiment showed, some excerpt use similar words.
These shared words might be related to the era the book was written in.
The following experiment try to investigate on this.

In the St-Jean dataset publication paper, the publication dates of each excepts are available~\cite{st_jean}.
First the publication date distribution of the dataset must be understood.
Figure~\ref{fig:dates_distribution} show the distribution of the publication date in the St-Jean dataset.
The date difference distribution for each pair of documents can be computed, Figure~\ref{fig:dates_differences_all} showcase this.
The average date difference in the dataset is 28.24 years with a standard deviation of 20.73 years.
Since this dataset contain multiple excerpt from the same book, considering only the links of different authors (false links) is preferable.
Figure~\ref{fig:dates_differences_false} shows the date difference distribution of the false links.
As expected the mean increased from 28.24 years to 29.04 years since there are fewer links with small date difference.
Same authors links (true links) are displayed in Figure~\ref{fig:dates_differences_r_true}, they confirm the previous statement that most of the same authors links have a low date difference with a mean of 5.11 years.

\begin{figure}
  \centering
  \caption{Dates and Date differences denstiy distributions on St-Jean for every excerpt}

  \subcaption{Date distribution}
  \label{fig:dates_distribution}
  \includegraphics[width=\linewidth]{img/dates_distribution.png}

  \subcaption{Date difference denstiy distributions}
  \label{fig:dates_differences_all}
  \includegraphics[width=\linewidth]{img/dates_differences_all.png}
\end{figure}

\begin{figure}
  \centering
  \caption{Date difference denstiy distributions in St-Jean}

  \subcaption{False links}
  \label{fig:dates_differences_false}
  \includegraphics[width=\linewidth]{img/dates_differences_false.png}

  \subcaption{True links}
  \label{fig:dates_differences_r_true}
  \includegraphics[width=\linewidth]{img/dates_differences_r_true.png}

  \subcaption{Top-r false links using a rank list with 85\% average precision.}
  \label{fig:dates_differences_r_false}
  \includegraphics[width=\linewidth]{img/dates_differences_r_false.png}
\end{figure}

The previous statistics can be compared to Figure~\ref{fig:dates_differences_r_false} which show the date difference density on the top-r false links (670 in case of St-Jean) on a rank list with 85\% average precision (z-score fusion of the retained text representations, Table~\ref{tab:9rl_results_st_jean_A_B}).
Two interesting information can be extracted here.
First the mean is lower by 7.75 years (29.04 - 21.29) compared to the false links and have a narrow standard deviation distribution which clearly indicate the importance of publication date in the ranking of the documents.
Secondly we can observe a drop after 35 years of date difference, which indicate that links in the interval $\left[0-35\right]$ years are harder to discriminate between a true link and a false link than links outside this interval.
This 35 years interval can be related to the generation factor, the age of woman giving birth is around 25-34 in France~\cite{generations}, the birth country of the authors in this dataset.
Each new generation tend to use its own vocabulary, and thus it can be harder to discriminate the author of text belonging to the same generation, if we assume that the authors write their books at around the same age.
In the other hand having different vocabulary can indicate a different time period and is often used to detect document forgery~\cite{savoy_stylo}.

\subsubsection{Distances matrix visualization}

When computing the rank lists, each link have its distance calculated and can be represented in a matrix for the non-comutative distances functions and in a symmetric matrix for the comutative distances functions.
In this matrix each document represant a row and a column.
To visualize represent the distances matrix and each element of the matrix is mapped to a pixel in an 2D image.
The element value in the matrix is mapped to the pixel bightness, the low values are in light colors and high values in dark colors.
A perfect distance matrix should have each same author documents pair with a low distances (light color in the image) and different authors documents pairs with a high distance (dark color in the image).
The greater the contrast between the true links (same author documents) and the false links (different authors documents) is, the better the text representation is, since the related rank list will have a greater average precision.
When the documents are sorted alphabetically by their authors and if the distances matrix can represent correctly the author style, same authors documents have generally a low distance, this creates light color squares in the diagonal.
The square size is related to the number of document written by this author.
The diagonal is the lightest color (white), since the distance between two same document is always 0, with respect to the identity of indiscernibles axoim of the distance functions.

The distance matrix for the best retained representation (highest average precision) and the worse retained text representation (lowest average precision) is visually presented in Figure~\ref{fig:distances_matrix_oxquarry} for Oxquarry.
Respectively the Clark distance on the $750$-MFW which gives an average precision of $0.89$, and the Tanimito distance on the $750$-MFW gives an $0.63$ average precision.
The diagonal is white in both images, light color square in the diagonal can clearly be observed on both distance matrix visualizations.
Clark have overall a good distances matrix, except some document pairs from Conrad, Hardy and Orczy which have darker colors, these are  ranked under the some false links in the rank list.
For Tanimoto, firstly one can observe these stripes, which are due to the max function in its computation, which create more cleaved decision in the score value.
The main suspects for this observation is probably due to some high frequency term only used in some excerpts, thus when normalizing by the sum of maxima in the Tanimoto computation, document with these high relative frequency create these stripes.

\begin{figure}
  \caption{Distance matrix visualization Oxquarry}
  \label{fig:distances_matrix_oxquarry}

  \caption{Best retained text representation for Oxquarry ($750$-MFW tokens with Clark)}
  \label{fig:distance_matrix_oxquarry_clark}
  \includegraphics{img/distance_matrix_oxquarry_clark.png}

  \caption{Worse retained text representation for Oxquarry ($750$-MFW tokens with Tanimoto)}
  \label{fig:distance_matrix_oxquarry_tanimoto}
  \includegraphics{img/distance_matrix_oxquarry_tanimoto.png}
\end{figure}
