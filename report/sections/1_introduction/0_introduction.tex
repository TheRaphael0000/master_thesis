% The Introduction chapter contains: theproblem statement (which scientific questions have to be answered), objectives (which goals have to be achieved) and Outline (which research methods are taken in account).

\section{Introduction \label{sec:introduction}}

journalism/law ~\cite{pan16_clustering_site}
internet forums/blogs anonymous users ~\cite{kocher_pan16}
twitter fake news ~\cite{unine_pan20_fake}
forensic inverstigation cyber-criminals
cybersecurity ~\cite{automated_unsupervised}
e-mail spam filter ~\cite{savoy_probability}

\subsection{Research questions}

The main research question of this thesis is:

\begin{itemize}
  \item Using the principle of combination of evidence can the rank lists' fusion improve the quality of the rank lists ?
\end{itemize}

By answering this question, this provides a simple framework that can be used in multiple fields using rank lists such as the information retrieval, recommender systems, authorship attribution, authorship clustering and most classification problems based on complete graphs.

\subsection{Contributions}

In this study, the following contributions are made to the scientific community:

\begin{enumerate}
  \item An evaluation of most frequent words frequency feature extraction strategy using : tokens, n-grams, POS n-grams text representation and multiple distance metrics.
  \item An brief evaluation of the compression based methods for authorship attribution.
  \item A methodology to create robust rank lists for authorship attribution and authorship clustering, using combination of evidence based on rank lists fusions.
  \item An brief evaluation of the mean Silhouette score maximization for unsupervised authorship clustering as well as an optimization proposition for authorship clustering.
  \item A proposition and evaluation of a semi-supervised authorship clustering model based on an authorship attribution modelling using a mixture of two beta distribution.
  \item A proposition and evaluation of a supervised authorship clustering model based logistic regression.
\end{enumerate}

\subsection{Overview}

An overview of the current technique used in the authorship field is presented in Section~\ref{sec:state_of_the_art}.
Section~\ref{sec:definitions_and_corpora} show some definitions and introduce the corpora used for the evaluations part.
In Section~\ref{sec:methods} the methods used in the current study are described, this section encompasses methods proposed such as the rank generations, rank lists fusions and the clustering proposed methods.
The methods proposed in the previous are evaluated in Section~\ref{sec:evaluation} as well as the results of diverse experiments.
Finally, Section~\ref{sec:conclusion} conclude this paper by reviewing the experiments, results and indicate the potential clues to continue the study.
