% The Introduction chapter contains: the problem statement (which scientific questions have to be answered), objectives (which goals have to be achieved) and Outline (which research methods are taken in account).

\chapter{Introduction \label{sec:introduction}}

In a world where the information starts to propagate fast through social networks, forums or blogs and allow anonymous communication or pseudonymous communications, security and authentication problems can occur \cite{automated_unsupervised} \cite{kocher_pan16}.

When internet crimes take place, countries security organizations work together with ISPs and websites to trace criminals using network clues.
But technologies, such as : onion routing and proxies, can even provide to attackers an anonymity on a network level.
Due to the increase of criminal using anonymity network-based system, new algorithm able to detect authors solely based on the text are needed \cite{automated_unsupervised} \cite{attribution_in_cyberspace}.

The field of stylometry consists in analyzing the style of a written text.
This can be used to identify the authorship or to draw the author profile.
These techniques can be used for journalists, law courts, cybersecurity and forensic investigators to give authorship clues for their work when dealing with completely anonymized texts, e.g.: ransom notes, email threats, blog posts or cybercriminals source code comments \cite{pan16_clustering_site}.
In the current study, fake news spreading or phishing by e-mails attempts can be detected using stylometry \cite{unine_pan20_fake_news}.

Developing automatic and unsupervised authorship techniques is required to solve these problems \cite{automated_unsupervised}.
A typical unsupervised authorship problem is to find in a large corpus of texts which are written by the same author.
This problem is called authorship clustering.
An example where the authorship clustering can be useful is to be able to group text written by anonymous writers (e.g. group of terrorists or old documents from unknown authors).
For this study, stylometric, compression, authorship verification and fusion techniques are applied to the authorship clustering problem.

\subfile{1_research_questions}
\subfile{2_contributions}
\subfile{3_implementation}
\subfile{4_overview}
