\chapter{Authorship Clustering \label{sec:clustering}}

In this chapter, authorship clustering methods will be presented and evaluated.

As explained in Section~\ref{sec:def_clustering}, authorship clustering aim to regroup documents into clusters.
The best achievable clustering have every same author document in the same cluster and every different author document in a different cluster.
The quality can be measured with the metrics presented in Section~\ref{sec:clustering_evaluation_meterics}.

For the clustering part, multiple methods based on hierarchical clustering are proposed and evaluated:
\begin{itemize}
  \item
  Silhouette-based clustering: A fully unsupervised clustering which aims to minimize the intra-cluster distances and maximize the nearest cluster distance.
  \item
  Distribution-based clustering: A supervised clustering technique which aims to model known corpus rank lists and compute a decision point.
  \item
  Regression-based clustering: A supervised clustering technique which takes clues from previous corpora analysis to lead the clustering.
\end{itemize}

The hierarchical clustering is the simplest connectivity model to achieve clustering from a distance matrix.
As explained in Section~\ref{sec:distances_matrix}, rank lists and distance matrices are related, thus after computing rank lists this clustering method seems to us to be the right choice.
Additionally the hierarchical clustering model can use any distance measure which allow flexibility.

The main draw back of this method is the computation cost to compute the rank lists / distance matrices ($O(n^2)$).
This problem is partially ignorable since this study focus on texts which are easily to handle, for example when compared to images.

\subfile{1_hierarchical_clustering}
\subfile{2_silhouette-based_clustering}
\subfile{3_distribution-based_clustering}
\subfile{4_regression-based_clustering}
\subfile{5_clustering_summary}
