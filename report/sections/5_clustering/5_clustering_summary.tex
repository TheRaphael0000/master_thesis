\subsection{Clustering Summary}

Table~\ref{tab:clustering_evaluation_summary} contains a summary of the proposed method to evaluate using the $B^3_{F_1}$ score and the $r_{diff}$.

Each method produces similar results, but the distribution-based clustering with complete linkage give the best results, $B^3_{F_1} = 0.82$.
When compared to the Silhouette-based methods and the regression-based method produce slightly least accurate results.
The non-tweak Silhouette-based model give the least accurate results.

Since the single linkage criterion show terrible results for every models, this criterion is left.
The complete linkage and average linkage give overall the same results.
Though the average precision have better results on 3 out of the 4 models, the distribution based model with the complete linkage give the best results across the board.

The tweaked Silhouette-based model with average linkage and complete linkage and the distribution-based with average linkage are the best models to estimate the number of clusters, $r_{diff} = 0.06$.

Overall the results are close to the upper bound.

\begin{table*}[t]
  \centering
  \caption{Mean retained rank lists $B^{3}_{F_1}$/$r_{diff}$ for each corpus pair}
  \label{tab:clustering_evaluation_summary}
  \begin{tabular}{l c c c}
    \toprule
    Clustering method                  & Single Linkage & Average Linkage    & Complete Linkage \\
    \midrule
    Silhouette-based ($\alpha = 0$)    & 0.74/0.22      & \textbf{0.76/0.17} & 0.75/0.17 \\
    Silhouette-based ($\alpha = -0.2$) & 0.79/0.08      & \textbf{0.81/0.06} & 0.80/0.06 \\
    Distribution-based                 & 0.22/0.16      & 0.76/0.06          & \textbf{0.82/0.07} \\
    Regression-based                   & 0.64/0.10      & \textbf{0.80/0.13} & 0.74/0.19 \\
    \midrule
    \textit{Upper bound}           & \textit{0.83/0.00} & \textit{0.88/0.00} & \textit{0.87/0.00} \\
    \bottomrule
  \end{tabular}
\end{table*}
