\section{Silhouette-based clustering\label{sec:silhouette-based_clustering}}

\subsection{Method}

The main idea of the Silhouette-based clustering is to evaluate in an unsupervised manner the clustering result for each number of clusters (at each merge step of the hierarchical clustering).
When discarding the clustering with $N$ and clusters $1$, this produce $N - 2$ possible clustering, each of those are evaluated using the unsupervised mean Silhouette score metric.
The mean Silhouette score is defined in Definition~\ref{def:silhouette}.

\begin{definition}[Mean Silhouette score~\cite{sklearn}~\cite{wiki_silhouette}]
  \label{def:silhouette}
  The mean Silhouette score $s$ is an unsupervised clustering metric which evaluate a clustering result by measuring the cohesion $a(i)$ and separation of the clusters $b(i)$.

  \begin{gather*}
    s = \frac{1}{|C|} \sum_{i = 0}^{|C|} \frac{b(i) - a(i)}{max(a(i), b(i))}
  \end{gather*}
  \begin{gather*}
    \begin{split}
      a(i)&: \text{mean intra-cluster distance} \\
      a(i)& = \frac{1}{|C_i| - 1} \sum_{j \in C_i, i\neq j} d(i, j) \\
      b(i)&: \text{mean nearest-cluster distance} \\
      b(i)& = \min_{k\neq i} \frac{1}{|C_k|} \sum_{j \in C_k} d(i, j) \\
    \end{split}
  \end{gather*}
  With $C$ the set of clusters, $C_i$ the i-th cluster, $d(i, j)$ the distance between the document i and j.
  $d(i, j)$ are pre-computed and used for the rank list.

  The value is ranged between -1 and 1, a large value indicate a good cohesion and good separation of the clusters (low intra-cluster distance, high nearest-cluster distance).
\end{definition}

The right number of cluster is not known.
Using this technique, we suppose that the best number of clusters, is the one with the largest mean Silhouette score.
Each step of the hierarchical clustering is evaluated with the mean Silhouette Score.
Only the one with the largest is kept.

An alternative to this method is the Iterative Positive Silhouette (IPS) and was proposed in Layton, Watters, Dazeley (2011)~\cite{automated_unsupervised}.
This method uses the median Silhouette score instead, and use another stopping procedure based on the sign of the score.

\subsection{Evaluation}

For this experiment, the goal is to test the Silhouette-based hierarchical clustering method on the literature corpora.

The rank lists used for this experiment are the ones from the retained text representation (9 for St-Jean and 7 for Brunet and Oxquarry), see Table~\ref{tab:rls_oxquarry_brunet}~and~\ref{tab:rls_st_jean} in annex.

The evaluation methodology is presented in the schema in Figure~\ref{fig:clustering_evaluation_methodology}.
The $B^3_{F_1}$ score and $r_{diff}$ average on the Silhouette-based clustering are presented in Table~\ref{tab:silhouette-based_clustering}.
The complete table is available in Table~\ref{tab:silhouette-based_clustering_full} in annex.

The average r-ratio difference is a positive value ranging between $\left[0.17, 0.22\right]$ depending on the linkage criterion.
Having an r-ratio larger than $0$ indicate that the estimated number of cluster on every corpus is overestimated.
This means that the mean neareast-cluster distance is greater than the mean intra-cluster distance, even when dealing with the right number of clusters.
In Table~\ref{tab:silhouette-based_clustering}, the r-ratio is clearly too high, and the $B^3_{F_1}$ score is $\sim 14\%$ worse than the upper bound.

This can be due to the fact that the rank list used for the clustering is not perfect (AP $\neq 1$).

\begin{table}
  \centering
  \caption{Silhouette-based clustering evaluation (Maximal Silhouette, $\alpha = 0$), mean $B^3_{F_1}$/$r_{diff}$}
  \label{tab:silhouette-based_clustering}

  \resizebox{\linewidth}{!}{
  \begin{tabular}{l c c c}
    \toprule
           & \multicolumn{3}{c}{Linkage criterion} \\
    Corpus    & Single     & Average   & Complete \\
    \midrule
    Oxquarry  & 0.76/0.18 & 0.79/0.12 & 0.78/0.13 \\
    Brunet    & 0.69/0.28 & 0.71/0.26 & 0.73/0.23 \\
    St-Jean A & 0.59/0.33 & 0.64/0.24 & 0.61/0.26 \\
    St-Jean B & 0.91/0.08 & 0.91/0.06 & 0.90/0.06 \\
    \midrule
    Absolute mean & 0.74/0.22 & 0.76/0.17 & 0.75/0.17 \\
    \midrule
    \textit{Upper bound} & \textit{0.83/0.00} & \textit{0.88/0.00} & \textit{0.87/0.00} \\
    \bottomrule
  \end{tabular}
  }
\end{table}

\subsection{Tweak for Authorship Clustering}

We want to mitigate having the number of cluster overestimated by the Silhouette-based clustering method.
An easy solution is to use the clustering results produced earlier in the algorithm steps (less merges, less clusters, less overhestimation of the number of clusters).
This corresponds to a non-maximal value of the mean Silhouette score, on the left side of the maximal mean Silhouette score.
This procedure is a form of early stop.

In this study, we introduce a parameter called $\alpha$.
$\alpha$ represents a percentage to subtract to the maximal mean Silhouette score to obtain a new target with a lower mean Silhouette score (instead of the maximal value).

The mean Silhouette score across the number of clusters is a concave function.
The new target, can thus be either on the left or on the right of the maximal mean Silhouette.
To remove this ambiguity, the sign of $\alpha$ indicate on which side of the maximal Silhouette score the target should be.

With $\alpha = 0$, this corresponds to the maximal mean Silhouette score.
With a negative $\alpha$ the left side is targeted and with a positive $\alpha$, the right side.

The clustering result with the Silhouette score the closest to the target on the desired side is retained.

\begin{definition}[$\alpha$-Silhouette]
  The $\alpha$-Silhouette target score found by using the maximal Silhouette score as a baseline and is adjusted using a parameter called $\alpha$.
  \begin{gather*}
    \begin{aligned}
    target &= max(Scores) - |\alpha| \cdot max(Scores) \\
           &= max(Scores) \cdot (1 - |\alpha|)
    \end{aligned}
  \end{gather*}

  Notice, the sign of the $\alpha$ parameter is not taken into account to select the score target.
\end{definition}

Example~\ref{ex:alpha_correction} show the $\alpha$ computation and usage.

For example, by using $\alpha = -0.2$, we aim to correct this overshoot and increase the quality of the clustering results.
This value was chosen by grid search to optimize the $B^3_{F_1}$.

Table~\ref{tab:silhouette-based_clustering_alpha} show the evaluation with $\alpha = -0.2$, the complete table is available in Table~\ref{tab:silhouette-based_clustering_alpha_full} in annex.

With this correction technique, in average the $r_{diff}$ is closer to 0 and the average $B^3_{F_1}$ is increased in average by $7\%$ across all the corpora.
With $\alpha = -0.2$, we obtained a $B^3_{F_1}$ only $8\%$ worse than the upper bound, instead of the $14\%$ obtained with $\alpha=0$.
The average linkage criterion give the best results for this clustering method.
The $B^3_{F_1}$ for the average linkage criterion is in average $2\%$ better than the other criterions.

\begin{example}
  \centering
  \caption{$\alpha$ correction}
  \label{ex:alpha_correction}

  \subcaption{Silhouette Scores for each number of clusters}
  \begin{tabular}{r r}
    \toprule
    Number of clusters & Silhouette Score \\
    \midrule
    3 & 2.5 \\
    4 & 3.2 \\
    5 & 3.5 \\
    6 & 3.9 \\
    7 & 3.1 \\
    8 & 2.9 \\
    \bottomrule
  \end{tabular}

  \vspace{0.5cm}

  \subcaption{$\alpha$ computations}
  \raggedright
  with $max(Scores) = 3.9$ and $\alpha = -0.2$

  \begin{gather*}
    \begin{aligned}
    target &= 3.9 \cdot (1 - |-0.2|) \\
           &= 3.9 \cdot 0.8 \\
           &= 3.12
     \end{aligned}
   \end{gather*}

  \vspace{0.5cm}

  \subcaption{Clustering selection}
  Since $\alpha$ is negative, the left side of the maximal mean Silhouette is used (negative: smaller number of clusters).

  Thus, the clustering on the left side with the closest Silhouette score to the target is the one with 4 clusters (score: $3.2$).

  With $\alpha = 0.2$, the target would be the same, but the right side would be selected instead.
  The number of cluster selected would be 7.

  With $\alpha = 0$, the number of cluster selected would be 6, since the target would be $3.9$.
\end{example}

\begin{table}
  \caption{Silhouette-based clustering evaluation ($\alpha = -0.2$), mean $B^3_{F_1}$/$r_{diff}$}
  \label{tab:silhouette-based_clustering_alpha}

  \resizebox{\linewidth}{!}{
  \begin{tabular}{l c c c}
    \toprule
           & \multicolumn{3}{c}{Linkage criterion} \\
    Corpus    & Single     & Average   & Complete \\
    \midrule
    Oxquarry  & \textbf{0.81/0.06} & 0.78/0.03 & 0.79/0.02 \\
    Brunet    & 0.78/0.10 & \textbf{0.80/0.09} & \textbf{0.80/0.10} \\
    St-Jean A & 0.70/0.14 & \textbf{0.77/0.08} & 0.76/0.09 \\
    St-Jean B & 0.86/0.02 & \textbf{0.87/0.03} & 0.86/0.03 \\
    \midrule
    Absolute mean & 0.79/0.08 & \textbf{0.81/0.06} & 0.80/0.06 \\
    \midrule
    \textit{Upper bound} & \textit{0.83/0.00} & \textit{0.88/0.00} & \textit{0.87/0.00} \\
    \bottomrule
  \end{tabular}
  }

  \vspace{0.2cm}

  \textit{In bold: For each corpus, the criterion with the largest $B^3_{F_1}$}
\end{table}
