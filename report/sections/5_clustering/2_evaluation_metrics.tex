\subsection{Authorship clustering evaluation metrics}

To evaluate clustering, the metrics used are the BCubed~\cite{bcubed} family (sometimes called $B^3$, in this report).
BCubed has shown to satisfy the four following important constraints when evaluating clusterings~\cite{bcubed}~:

\begin{enumerate}
  \item \textit{Cluster Homogeneity} (different authors should be in the different clusters)
  \item \textit{Cluster Completeness} (same authors should belong to the same cluster)
  \item \textit{Rag Bag constraint} (noisy / miscellaneous authors should be in the same cluster and not in \textit{healthy} clusters)
  \item \textit{Cluster size vs quantity constraints} (favour large cluster)
\end{enumerate}

These metrics are used in the clustering task at PAN @ CLEF~\cite{pan16}.
In addition to the BCubed family, this study introduce a simple metric for evaluating the clustering, which is called the Cluster difference.

\begin{definition}[Correctness~\cite{bcubed}]
  The $BCubed$ metric family is based on the following \textit{Correctness} principle.
  Let L(e) and C(e) be the author and the cluster of an element e.
  The Correctness is following the biconditional condition on the author and cluster equality (biconditional: $A \Longleftrightarrow B \equiv (A \land B) \lor (\neg A \land \neg B)$).
  \begin{gather*}
    Correctness(e, e') = \\
    \begin{cases}
      1, & if (L(e) = L(e')) \Longleftrightarrow (C(e) = C(e'))\\
      0, & otherwise
    \end{cases}
  \end{gather*}
  In other terms, the correctness has a value of 1 (100\% correct) if the two elements are in the both in the same cluster and has the same author OR both in a different cluster and a different author.
\end{definition}

\begin{definition}[$BCubed$ Precision~\cite{bcubed}]
  The $BCubed$ Precision correspond to the average of correctness for all elements on the average of all element such that \textbf{their clusters are the same}.
  \begin{equation}
    B^3_{precision} = \text{Avg}_{e}[\text{Avg}_{e' C(e)=C(e')}[Correctness(e, e')]]
  \end{equation}
\end{definition}

\begin{definition}[$BCubed$ Recall~\cite{bcubed}]
  The $BCubed$ Recall correspond to the average of correctness for all elements on the average of all element such that \textbf{their authors are the same}.
  \begin{equation}
    B^3_{recall} = \text{Avg}_{e}[\text{Avg}_{e' L(e)=L(e')}[Correctness(e, e')]]
  \end{equation}
\end{definition}

\begin{definition}[$BCubed F_1$ Score~\cite{bcubed}]
  $BCubed F_1$ Score uses the harmonic mean between the $B^3_{precision}$ and $B^3_{recall}$.
  \begin{equation}
    B^3_{F_1} =
    2 \cdot \frac{B^3_{precision} \cdot B^3_{recall}}
    {B^3_{precision} + B^3_{recall}}
  \end{equation}
  The $F_\beta$ measures, not shown here, provide a parametric way to represent with a single value the two counterbalancing measures in this case, the $B^3_{F_1}$ is computed using the $F$ measures with $\beta = 1$ and the $B^3_{precision}$ and $B^3_{recall}$ scores.
\end{definition}

\begin{definition}[Cluster difference]
  This metric aim to evaluate if the clustering found the right number of cluster.
  The cluster difference is the number of cluster found $p$ minus the actual number of cluster $k$ (number of distinct authors in the corpus).
  \begin{equation}
    Cluster_{diff} = p - k
  \end{equation}
  A positive value indicates an overestimation of the real number of cluster, a negative value indicate the underestimation.
  Zero indicate that the right number of cluster was found.
  This value can also be used to summarize if the $B^3_{recall}$ is greater or not than the $B^3_{precision}$.
  A positive Cluster diff should indicate a larger $B^3_{precision}$ than $B^3_{recall}$, and vice versa.
  This value can be normalized by the number of documents N, which correspond to the difference of the r ratios.
  This is useful when comparing corpora results with different number of clusters.
  \begin{equation}
    r_{diff} = \frac{p}{N} - \frac{k}{N} = \frac{p - k}{N}
  \end{equation}
  As stated in the PAN16 evaluation campaign paper, estimating correctly the number of clusters and the r ratio is a first step to produce a good clustering~\cite{pan16}.
\end{definition}
