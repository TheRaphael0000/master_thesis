\section{Fusion summary}

The Z-Score fusion and regression fusion was shown to improve rank lists quality when fusing \textit{good} rank list together.

The rank lists obtained by fusing every retained rank lists with the Z-Score method are referred as \textit{Z-Score rank lists}.
This corresponds to one rank list for each corpus.

When fusing rank lists with the regression fusion, pairs of corpora are required (one for training and one for testing).
When evaluating this method with four corpora (Oxquarry, Brunet, St-Jean A and St-Jean B), this creates $4^2 = 16$ possible corpus pairs which yield one rank list each.

These two methods are summarized in the schema in Figure~\ref{fig:schema-fusion}.

Table~\ref{tab:fusions_scores} in annex contain the rank lists' evaluation, for every corpus, with every retained rank lists for the fusion.
The average precision obtained with the Z-Score rank lists is greater than each individual methods for every corpus except for the Oxquarry corpus.
The Oxquarry Z-Score rank list have an average precision of 0.84 but using the individual methods: \textit{the Cosine distance or the Clark distance with the $750$-MF tokens} gives an average precision of 0.89.
The corpus with the worst average precision using the Z-Score rank lists is obtained on Brunet with 0.76 and is as efficient as the individual \textit{BZip2 compression with CBC distance} method.
Even though here the Z-Score rank list have lower quality than an individual method, in an unsupervised environment this can not be detected (average precision require labels).
Thus, for unsupervised tasks, we advise to use the Z-Score fusion since it provides in average better results than each individual methods.

Table~\ref{tab:fusion_evaluation_summary} is a summary of the results for the fusion using every individual methods.
For both methods, the mean average precision across all corpora reach 0.85.
The mean average precision for the fused rank lists is 7\% better using the fusion than the individual methods average.
Since the regression fusion require a training corpus and had shown some weakness during its evaluation.
We recommend using the Z-Score fusion instead.

The proposed veto methods did not provide good results and are not advised to use.

\begin{figure*}
  \centering
  \caption{Fusion methods schema}
  \label{fig:schema-fusion}
  \includegraphics[width=1\linewidth]{img/schema-fusion.png}
\end{figure*}

\begin{table}
  \caption{Fusion evaluation summary, mean across every corpora}
  \label{tab:fusion_evaluation_summary}
  \resizebox{\linewidth}{!}{
  \begin{tabular}{l c c c}
    \toprule
                      & AP   & RPrec & HPrec \\
    \midrule
    Z-Score fusion    & 0.85 & 0.78   & 101.50 \\
    Regression fusion & 0.85 & 0.78   & 98.50  \\
    \midrule
    \textit{Individual methods average} & \textit{0.79} & \textit{0.71} & \textit{78.88} \\
    \bottomrule
  \end{tabular}
  }
\end{table}
