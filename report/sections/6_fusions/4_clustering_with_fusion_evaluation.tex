\subsection{Clustering with Fusions Evaluation}

Table~\ref{tab:silhouette-based_clustering_zscore}~/~\ref{tab:silhouette-based_clustering_zscore_alpha}~/~\ref{tab:distribution-based_clustering_zscore}~/~\ref{tab:regression-based_clustering_zscore} in annex show the evaluation of the clustering task for rank lists obtained by fusing every retained rank lists using the Z-Score method.

Table~\ref{tab:clustering_zscore_evaluation_summary} is a summary of the results.
The upper bound is, using the same rank lists, the best clustering achievable with the hierachical clustering and a specific linkage criterion.
It is obtained by evaluating the clustering a each step, the ones with the greatest metrics is kept.

\begin{table}[t]
  \centering
  \caption{Retained rank lists Mean $B^{3}_{F_1}$/$r_{diff}$ for each corpus pair}
  \label{tab:clustering_zscore_evaluation_summary}
  \resizebox{\linewidth}{!}{
  \begin{tabular}{l c c}
    \toprule
                                       & \multicolumn{2}{c}{Linkage criterion} \\
    Clustering method                  & Average             & Complete \\
    \midrule
    Silhouette-based ($\alpha = 0$)    & \textbf{0.75/0.16}  & \textbf{0.75/0.16} \\
    Silhouette-based ($\alpha = -0.2$) & \textbf{0.87/0.07}  & 0.84/0.08 \\
    Distribution-based                 & 0.83/-0.02          & \textbf{0.86/0.04} \\
    Regression-based                   & \textbf{0.84/0.07}  & 0.80/0.13 \\
    \midrule
    \textit{Rank lists upper bound}    & \textit{0.92/0.00} & \textit{0.89/0.00} \\
    \bottomrule
  \end{tabular}
  }
\end{table}
