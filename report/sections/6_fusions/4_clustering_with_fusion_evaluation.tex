\section{Clustering with Fusion Evaluation}

Table~\ref{tab:silhouette-based_clustering_zscore}~/~\ref{tab:silhouette-based_clustering_zscore_alpha}~/~\ref{tab:distribution-based_clustering_zscore}~/~\ref{tab:regression-based_clustering_zscore} in Annex show the evaluation of the clustering task for rank lists obtained by fusing every retained rank lists using the Z-Score method.

Table~\ref{tab:clustering_zscore_evaluation_summary} is a summary of the results.
The upper bound is, using the same rank lists, the best clustering achievable with the hierarchical clustering for specific linkage criterion.
It is obtained by evaluating the hierarchical clustering at each step of the algorithm.
The clustering result with the greatest metrics is the upper bound for the hierarchical clustering.

With the Z-Score fusion the upper bound rise from $0.88$ to $0.92$ with the average linkage criterion, this corresponds to a $\sim 5\%$ relative increase.
For the complete linkage, the $B^{3}_{F_1}$ rise from $0.87$ to $0.89$, which correspond to a $\sim 2\%$ relative increase.
With the Z-Score fusion rank lists, every clustering models have its results improved over the individual approaches.
This result further motivate the rank list fusion usefulness.

The Silhouette-based clustering method with the average linkage criterion yield the best results across the four models.
This result is only slightly better than the distribution-based model with the complete linkage criterion.
This model was the one that obtained the best results for the clustering on the individual rank lists methods.

For the individual methods, the average linkage criterion is the best criterion for each method, except the distribution-based clustering, which obtain better results with the complete linkage.

The $B^{3}_{F_1}$ achieved for each model with the best linkage criterion is $5$ to $10\%$ worse than the upper bound, if we exclude the Silhouette-based model (with $\alpha = 0$) which is $18\%$ worse than the upper bound.
Excluding the non-tweak Silhouette-based model clustering, every other proposed clustering model seem to give reasonable results with fused rank lists.

The clustering with the fused rank list is in average $6\%$ better than the clustering with the individual methods.

\begin{table}[t]
  \centering
  \caption{Retained rank lists Mean $B^{3}_{F_1}$/$r_{diff}$ for each corpus pair}
  \label{tab:clustering_zscore_evaluation_summary}
  \resizebox{\linewidth}{!}{
  \begin{tabular}{l c c}
    \toprule
                                       & \multicolumn{2}{c}{Linkage criterion} \\
    Clustering method                  & Average             & Complete \\
    \midrule
    Silhouette-based ($\alpha = 0$)    & \textbf{0.75/0.16}  & \textbf{0.75/0.16} \\
    Silhouette-based ($\alpha = -0.2$) & \textbf{0.87/0.07}  & 0.84/0.08 \\
    Distribution-based                 & 0.83/-0.02          & \textbf{0.86/0.04} \\
    Regression-based                   & \textbf{0.84/0.07}  & 0.80/0.13 \\
    \midrule
    \textit{Rank lists upper bound}    & \textit{0.92/0.00} & \textit{0.89/0.00} \\
    \bottomrule
  \end{tabular}
  }
\end{table}
