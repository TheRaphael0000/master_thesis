\subsection{Rank lists}

Rank lists are used to order objects such that the most interesting object is at the top and every subsequent object become less interesting.
In information retrival systems, rank lists are used to order the results from the most relevant to the user's query to the least relevant.

\begin{definition}[Ranked list for authorship verification]
  A ranked list is a ordered list containing pairs of document.
  In most cases, the rank list contain every possible pairs of documents (every links).
  The ranked list is generally ordered by a score which can be either be ascendant in the case of scoring based on distance metrics and descendant when using similarity metrics.
  \begin{equation}
    L = (((X_a, X_b), Score(X_a, X_b)) | X_a \neq X_b \forall (X_a, X_b))
  \end{equation}
  \begin{equation}
    |L| = \frac{N \cdot (N - 1)}{2}
  \end{equation}
\end{definition}

This method is not the most efficiant, since it needs to compute the distance function for pairs of vectors ($\frac{n \cdot (n - 1)}{2}$) and thus have computetional and space complexity of $O(n^2)$.
Table~\ref{tab:rank_list_example} show a simple example using 2D points and the manhattan distance (presented in Section~\ref{sec:fv_distances}) to create a rank list.

\begin{table}
  \centering
  \caption{Rank list example using 2D points and the manhattan distance}
  \label{tab:rank_list_example}

  \subcaption{List of 2D points}
  \begin{tabular}{l r r}
    \toprule
    Point & X & Y \\
    \midrule
    0 & 0 & 0 \\
    1 & 1 & 2 \\
    2 & 4 & 6 \\
    3 & -1 & -4 \\
    \bottomrule
  \end{tabular}

  \subcaption{Pairwise manhanttan distances}
  \begin{tabular}{l l}
    \toprule
    Pair & Distance \\
    \midrule
    (0, 1) & $|0-1| + |0-2| = 3$ \\
    (0, 2) & $|0-4| + |0-6| = 10$ \\
    (0, 3) & $|0+1| + |0+4| = 5$ \\
    (1, 2) & $|1-4| + |2-6| = 7$ \\
    (1, 3) & $|1+1| + |2+4| = 8$ \\
    (2, 3) & $|4+1| + |6+4| = 15$ \\
    \bottomrule
  \end{tabular}

  \subcaption{Ordered rank list}
  \begin{tabular}{l c r}
    \toprule
    Rank & Pair & Distance \\
    \midrule
    0 & (0, 1) & $3$ \\
    1 & (0, 3) & $5$ \\
    2 & (1, 2) & $7$ \\
    3 & (1, 3) & $8$ \\
    4 & (0, 2) & $10$ \\
    5 & (2, 3) & $15$ \\
    \bottomrule
  \end{tabular}
\end{table}
