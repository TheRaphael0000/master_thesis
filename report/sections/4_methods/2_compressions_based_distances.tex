\subsection{Compression-based distances \label{sec:compression_based_distances}}

This section covers the compression based strategy to compute a distance between two documents.
The main idea is to compress two documents A, B, compress the string concatenation AB of A, B and compare their sizes after compression using one of the distance measure demonstrated below.
Commonly a lossless compression algorithm is used, such as the Lempel-Ziv family (GZIP), block sorting family (BZip2), statistical family (PPM)~\cite{comparing_compression}.
This technique is based on the fact that compression algorithms tries to lower the Shannon entropy of a document, thus when compressing a document with a large Shannon entropy the compressed document should have a larger size after compression than a document with a small Shannon entropy.
When concatenating two documents with a lot in common, the entropy of the concatenated document should be lower than if the two documents are very different.
This approach has the benefit can produce rank lists in a nearly parameterless manner, only a distance metric and compression algorithm is needed.
A main drawback with this technique is the fact that the results can not be properly explained.

In this study the lossless compression algorithm used are : GZip, BZip2, LZMA. These algorithm are already implemented in the python standard library~\cite{python_standard_library}.
Each compression algorithm can be tweaked with different parameters, the default settings are used except for the compression level (trade-off compression time and compression size).
By setting the compression level to the maximal setting, this ensure that the produced file will have the lowest possible Shannon entropy reachable with this algorithm, thus providing the best possible approximation of distance when used with the compression distances.

\begin{definition}[Conditional complexity of compression~\cite{savoy_stylo}~\cite{comparing_compression}]
  The conditional complexity of compression of two documents A and B using the size after compression C is computed as follow:
  \begin{equation}
    CCC(A, B) = C(AB) - C(A)
  \end{equation}
  This metrics is not easy to use since the order of magnitude is not bounded and can depend a lot on the text sizes.
  The following ones try to mitigate this problem.
\end{definition}

\begin{definition}[Normalized compression distance~\cite{savoy_stylo}~\cite{comparing_compression}]
  The normalized compression distance of two documents A and B using the size after compression C is computed as follow:
  \begin{equation}
    NCD(A, B) = \frac{C(AB) - \min(C(A), C(B))}{\max(C(A), C(B))}
  \end{equation}
  This metric gives a value in the range $\left[0, 1+\epsilon\right]$, with $\epsilon$ being a small positive value created by the imperfection of compression algorithms.
\end{definition}

\begin{definition}[Compression-based cosine~\cite{savoy_stylo}~\cite{comparing_compression}]
  The compression-based cosine of two documents A and B using the size after compression C is computed as follow:
  \begin{equation}
    CBC(A, B) = 1 - \frac{C(A) + C(B) - C(AB)}{\sqrt{C(A) \cdot C(B)}}
  \end{equation}
  This metric have the same properties as the cosine distance (Definition~\ref{def:cosine_dist}).
\end{definition}
