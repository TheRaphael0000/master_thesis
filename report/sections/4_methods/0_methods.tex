\section{Methods \label{sec:methods}}

For the authorship clustering problem, the method adopted was to compare document using the authorship linkings strategies presented in \textit{Evaluation of text representation schemes and distance measures for authorship linking}~\cite{kocher_verification}.
These method aim to compare the simlarity of documents by considering reccurent stylistic patterns.
Text patterns can be for example from either reccurant words or Part-Of-Speech (POS).
Another possible way is to compare similarities between document explored in this study is to use compression techniques.

To achieve a good clustering, the main objective is to have the most reliable authorship linking rank list possible.
To increase the quality of the rank list, the proposed method is to use a combination of multiple rank list using different strategies to form an optimistically better rank list.
By using the most diverse strategies, we believe it is possible to increase the rank list quality using the principle of combination of evidences.

For the clustering part, multiple methods are explored and proposed.
One method using a full unsupervised clustering, by minimizing intercluster distances and maximizing nearest cluster distance.
A semi-supervised clustering technique using prevouisly computed values as a decision point.
Lastly a supervised clustering technique which takes clues from previous corpora analysis to lead the clustering.

\subfile{1_preprocessing}
\subfile{2_most_frequant_words}
\subfile{3_compressions_based_distances}
\subfile{4_rank_lists}
\subfile{5_rank_lists_fusion}
\subfile{6_authors_clustering}
\subfile{7_pipeline}
