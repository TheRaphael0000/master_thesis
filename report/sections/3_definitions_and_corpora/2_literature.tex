\subsection{Literature corpus \label{sec:lit_corpus}}

Table~\ref{tab:lit_corpora} show corpus information and statistics on the corpora.
These corpora contain large texts with 10'000 tokens in averages.
These texts are of high quality (very low number of spelling errors), compared to other corpus since they come from published books.
The number of text per authors is relatively large, these corpora have a low $r$ value.

The Oxquarry corpus contain 52 excerpts from English novels from 9 different authors (Butler, Chesterton, Conrad, Forster, Hardy, Morris, Orczy, Tressel, Stevenson).
The complete list of novel and authors can be found in annex in Table~\ref{tbl:oxquarry_corpus}.
The Oxquarry corpus is already tokenized, which means every word and punctuation are already separated.

The Brunet french corpus contain exactly 4 excerpts of novels for each of the 11 authors for a total of 44 excerpts.
The complete list of novel and authors can be found in annex in Tables~\ref{tbl:brunet_corpus}.
Authors present in this corpus are : Balzac, Chateaubriand, Flaubert, Marivaux, Maupassant, \\
Proust, Rousseau, Sand, Vernes, Voltaire, Zola.
Brunet is already tokenized in two different ways : One using the actual tokens in the text, and another one using only lemma.

The Saint-Jean corpus was also used~\cite{unine_corpus}.
It contains 200 excerpts from 68 French novels written by 30 different authors during the XIX century~\cite{st_jean}.
St-Jean has a token, a lemma and a parts-of-speech representations.
Words orthography are corrected and standardized for example: M., Mr., Monsieur, monsieur.
Dates of publications of each excerpt are available for this corpus.
A complete list of novel and authors can be found in Tables~\ref{tbl:st_jean_corpus_1}/\ref{tbl:st_jean_corpus_2}/\ref{tbl:st_jean_corpus_3}/\ref{tbl:st_jean_corpus_4} in annex.
St-jean was created in such way that it can be spliced in two parts.
One part contain the first 100 texts and the other one, the 100 following, which are called respectively \textit{St-Jean Serie A} and \textit{St-Jean Serie B}.
Both parts approximately contain the same number of authors and the same number and documents per author.
Since this corpus have more documents than the other one, both whole and spliced representation are used in different scenario, the statistics of the two individual parts and the whole corpus is displayed in Table~\ref{tab:lit_corpora}.

\begin{table*}
  \centering
  \caption{General information and statistics on the literary corpora}
  \label{tab:lit_corpora}
  \begin{tabular}{l c c c c c c c c c}
    \toprule
    \textbf{Name} &
    \textbf{Lang.} &
    \textbf{Authors} &
    \textbf{Texts} &
    \textbf{r} &
    \textbf{True Links} &
    \textbf{Links} &
    \textbf{$tl_r$} &
    \textbf{Avg. \#Tokens} &
    \textbf{Avg. Token size} \\
    \midrule
    Oxquarry & EN & 9 & 52 & 0.173 & 160 & 1326 & 0.121 & 11650 & 3.819 \\
    Brunet & FR & 11 & 44 & 0.25 & 66 & 946 & 0.07 & 9778 & 4.013 \\
    St-Jean & FR & 30 & 200 & 0.15 & 670 & 19900 & 0.034 & 11533 & 3.928 \\
    St-Jean A & FR & 17 & 100 & 0.17 & 330 & 4950 & 0.067 & 11552 & 3.949 \\
    St-Jean B & FR & 19 & 100 & 0.19 & 258 & 4950 & 0.052 & 11513 & 3.907 \\
    \bottomrule
  \end{tabular}
\end{table*}
