\subsection{Definitions \label{sec:basic_notions}}

This section contain basic notions commonly used in the authorship attribution, verification and clustering.
Aswell as basic metric to evaluate simple properties in a corpus.

\begin{definition}[Document]
  A document $X_i$ is a ordered list of token. A token is a non-empty string. Example:
  \begin{equation}
    X_i = ("\text{the}", "\text{quick}", "\text{brown}", "\text{fox}", "\text{.}")
  \end{equation}
  To obtain tokens from a long string containing a non-tokenized document, a tokenizer i needed.
  A document correspond to a sample without feature selection in the data science field.
\end{definition}

\begin{definition}[Author]
  An author $Y_j$ is a string describing the author. Example:
  \begin{equation}
    Y_j = "\text{Zola}"
  \end{equation}
  Since this study is related to clustering, authors are also called clusters.
  Authors are the ground truth labels for the clustering task.
\end{definition}

\begin{definition}[Corpus]
  A corpus contain two lists $X$ and $Y$.
  $X$ contain is a list of documents $X_i$ of size $N$ and $Y$ a list of authors $Y_j$ of size $k$.
  \begin{gather}
    X = (X_1, X_2, X_3, X_{...}, X_N) \\
    N = |X|
  \end{gather}
  \begin{gather}
    Y = (Y_1, Y_2, Y_3, Y_{...}, Y_k) \\
    k = |Y|
  \end{gather}
  A corpus is also called a dataset in most data science problems.
\end{definition}

\begin{definition}[Text Authorship]
  The function $f$, is a surjective-only function which map every text $X_i$ to a single author $Y_j$
  \begin{equation}
    Y_j = f(X_i)
  \end{equation}
  The set of $\hat{Y}_a$ is the set of document written by $Y_a$.
  \begin{equation}
    \hat{Y}_a = \{X_i | f(X_i) = a\}
  \end{equation}
  \begin{equation}
    N = \sum_{j}^k |\hat{Y}_j|
  \end{equation}
  In authorship verification, the goal is to find this $f$ function, which can estimate the author based on new document and previous documents from this author and documents from other authors.
  For the clustering task, finding such function is also needed, with the exception that the exact author is not required.
  Instead, only knowing that two document are from the same author is required, see Definition~\ref{def:relevant_set}.
\end{definition}

\begin{definition}[Relevant set\label{def:relevant_set}]
  The relevant set $R$ contain every possible pairs of documents with the same authors.
  Links in this set are called \textit{true links} in this study.
  \begin{equation}
    R = \{(X_a, X_b)\ |\ \left( f(X_a) = f(X_b) \right) \land \left(X_a \neq X_b \right) \forall (X_a, X_b)\}
  \end{equation}
  The non-relevant set $R$ contain every \textit{false links}.
  All links are contained in $L$, the union of the relevant and non-relevant set.
  \begin{equation}
    \bar{R} = \{(X_a, X_b)\ |\ \left( f(X_a) \neq f(X_b) \right) \land \left(X_a \neq X_b \right) \forall (X_a, X_b)\}
  \end{equation}
  \begin{gather}
    R \cap \bar{R} = \emptyset \\
    L = \bar{R} \cup R \\
    |L| = \frac{N * (N-1)}{2}
  \end{gather}
  The authorship clustering task aim to estimate these sets.
\end{definition}

\begin{definition}[r ratio~\cite{pan16}]
  The r ratio is the ratio between the number of different authors $k$ and the number of documents $N$ in a given corpus.
  \begin{equation}
    r = \frac{k}{N}
  \end{equation}
  The inverse of the r ratio is equivalent to the mean number of documents per authors.
  \begin{equation}
    \frac{1}{r} = \frac{N}{k} = \frac{1}{k} \cdot \sum_{i} |\hat{Y}_j|
  \end{equation}
  If $r$ is close to $0$, most documents are written by different authors and there is a great density of true links.
  On this other hand, if $r$ is close to $1$, most of the document are written by a single authors and there are few true links.
\end{definition}

\begin{definition}[true links' ratio]
  The true links' ratio is the ratio between the number of true links $|R|$ and the number of links $|L|$ in a given corpus.
  This ratio is an alternative to the r ratio and is correlated for corpus having the same number of documents per authors.
  \begin{equation}
    tl_r = \frac{|R|}{|L|}
  \end{equation}
  The value range in the interval $\left[0, 1\right]$.
  The lower the true links' ratio is, the closer to the Singleton Cluster baseline the corpus is.
  The Singleton Cluster baseline consider every sample as a different label (respectively every documents is from a different author).
  With a large true links' ratio, the dataset can be estimated with a Single Cluster baseline, which consider every sample is in the same cluster (respectively every documents are written by the same author).
\end{definition}
