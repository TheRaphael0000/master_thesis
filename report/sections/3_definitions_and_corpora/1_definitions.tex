\section{Definitions \label{sec:definitions}}

This section contains basic definitions and metrics to evaluate simple properties of a corpus.

\begin{definition}[Document / Text]
  A document or a text $X$ is an ordered list of token.
  A token is a non-empty string.

  Example:
  \begin{gather*}
    X = ("\text{the}", "\text{quick}", "\text{brown}", "\text{fox}", "\text{.}")
  \end{gather*}
  To obtain tokens from a long string containing a non-tokenized document, a tokenizer is needed.
  A document correspond to a sample without feature selection in the data science field.
\end{definition}

\begin{definition}[Author]
  An author $Y$ is a string describing the author.

  Example:
  \begin{gather*}
    Y = "\text{Zola}"
  \end{gather*}
\end{definition}

\begin{definition}[Corpus]
  A corpus contain two lists $X$ and $Y$.
  $X$ contain is a list of documents $X_i$ of size $N$ and $Y$ a list of authors $Y_j$ of size $k$.
  \begin{gather*}
    X = (X_1, X_2, X_3, X_{...}, X_N) \\
    N = |X|
  \end{gather*}
  \begin{gather*}
    Y = (Y_1, Y_2, Y_3, Y_{...}, Y_k) \\
    k = |Y|
  \end{gather*}
  A corpus is also called a dataset for most data science problems.
\end{definition}

\begin{definition}[Authorship \label{def:authorship}]
  The function $f$, is a surjective-only function which map every text $X_i$ to a single author $Y_j$
  \begin{gather*}
    Y_j = f(X_i)
  \end{gather*}
  The set of $\hat{Y}_a$ is the set of document written by $Y_a$.
  \begin{gather*}
    \hat{Y}_a = \{X_i | f(X_i) = Y_a\}
  \end{gather*}
  \begin{gather*}
    N = \sum_{j}^k |\hat{Y}_j|
  \end{gather*}
  In authorship attribution, the goal is to find a function $\hat{f}$, an approximation of the $f$ function.
  $\hat{f}$ is found using documents with known authors.
  The function $\hat{f}$ can then estimate the author of a new document.
\end{definition}

\begin{definition}[Relevant set\label{def:relevant_set}]
  The relevant set $R$ contains every different pairs of documents with the same authors.
  Links in this set are called \textit{true links} in this study.
  \begin{gather*}
    R = \{(X_a, X_b)\ \\
    |\ \left( f(X_a) = f(X_b) \right) \land \left(X_a \neq X_b \right) \\
    \forall (X_a, X_b) \}
  \end{gather*}
  The non-relevant set $\bar{R}$ contains every \textit{false links}, documents pair with different authors.
  \begin{gather*}
    \bar{R} = \{(X_a, X_b)\ \\
    |\ \left( f(X_a) \neq f(X_b) \right) \land \left(X_a \neq X_b \right) \\
    \forall (X_a, X_b)\}
  \end{gather*}
  All links are contained in $L$, the union of the relevant set $R$ and non-relevant set $\bar{R}$.
  \begin{gather*}
    L = \bar{R} \cup R \\
    R \cap \bar{R} = \emptyset \\
    |L| = \frac{N * (N-1)}{2}
  \end{gather*}
\end{definition}

\begin{definition}[r ratio~\cite{pan16}]
  The r ratio is the ratio between the number of different authors $k$ and the number of documents $N$ in a given corpus.
  \begin{gather*}
    r = \frac{k}{N}
  \end{gather*}
  The inverse of the r ratio is equivalent to the mean number of documents per authors.
  \begin{gather*}
    \frac{1}{r} = \frac{N}{k} = \frac{1}{k} \cdot \sum_{j}^{k} |\hat{Y}_j|
  \end{gather*}
  If $r$ is close to $0$, most documents are written by different authors and there is a great density of true links.
  On this other hand, if $r$ is close to $1$, most of the document are written by a single authors and there are few true links.
\end{definition}

\begin{definition}[true links' ratio]
  The true links' ratio is the ratio between the number of true links $|R|$ and the number of links $|L|$ in a given corpus.
  This ratio is an alternative to the r ratio and is correlated for corpus having the same number of documents per authors.
  \begin{gather*}
    tl_r = \frac{|R|}{|L|}
  \end{gather*}
  The value range in the interval $\left[0, 1\right]$.

  The lower the true links' ratio is, the closer to the Singleton Cluster baseline the corpus is.
  The Singleton Cluster baseline considers every sample as a different label (aim to estimate: every document is from a different author).

  With a large true links' ratio, the corpus can be estimated with a Single Cluster baseline, which consider every sample is in the same cluster (aim to estimate: every document are written by the same author).
\end{definition}
