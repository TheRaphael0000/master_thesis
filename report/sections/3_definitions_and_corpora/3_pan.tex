\section{PAN @ CLEF 2016}

During PAN @ CLEF 2016 clustering campaign, a group of corpora is given to the participants~\cite{pan16_corpus}.
The corpora are separated in two parts: a training part where 18 clustering problems and solutions are available and a second part with 18 problems without solutions.
The problems are in three languages (English, Dutch, Greek) and two genres (articles and reviews)~\cite{pan16}.
Detailed statistics on the corpora can be found in Annex (Table~\ref{tab:pan_corpus}).

The $r$ ratio is closer to $1$ than $0$ for most of the problems, which indicate a rather larger number of single author clusters.
The baseline \textit{Singleton Cluster} (every document are considered in a different cluster) is a challenging problem to overcome for this dataset.

The mean number of tokens, for each problem in corpora, is in the range $[142-1533]$.
Compared to the literary datasets presented in Section~\ref{sec:lit_corpus} this corresponds to approximately 85\% to 98\% fewer tokens.

Additionally, the true link ratio is rather low for all problem which means that without a strong system, finding the correct true links is even harder, thus Singleton Cluster can give better results than most of the standard approaches.
This problematic does not promote researchers to come up with an efficient generic solution.
