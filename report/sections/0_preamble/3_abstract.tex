\begin{abstract}
  \thispagestyle{plain}
  \setcounter{page}{4}
  This study aim to select different methods, text representations and distance metrics to create rank lists for the authorship clustering problem.
  Among the vector based text representation: Words frequencies, Lemmas frequencies, Letters $n$-grams frequencies and POS sequences frequencies are used.
  The \textit{most frequent} strategy is applied to create document feature vectors.
  $L^1$, $L^2$ and inner product based distances are used in conjunction to compute distance between the vectors representations of the documents.
  In addition to the vector representation, compression techniques are also applied.
  The Oxquarry, Brunet and St-Jean corpora are used to evaluate the system.
  In a first part, the rank lists obtained are used to solve the authorship clustering task.
  The clustering task is achieved by using three different models based on hierarchical clustering.
  A Silhouette score based model, a distribution-based model and a regression-based clustering models are proposed and evaluated.
  The clustering obtained by these models are close to the best achievable clustering for the rank lists used.
  In a second part, a new rank list is created by fusing rank lists obtained using different strategies.
  Two approaches are explored for the fusion: one use Z-Score normalization and another one use logistic regression.
  The rank list obtained by fusion is shown to have better performances than the best rank list used for the fusion.
  In addition to the fusion, two veto strategy are proposed to try to enhance the fusion quality.
  The veto do not show any significant improvements with rank lists used.
  Lastly, rank lists obtained by fusion are used for the authorship clustering task.
  We showed that using rank lists obtained by fusion yield better clustering results than the ones obtained by individual methods.

  {\small \textbf{\textit{Keywords---}} Rank List, Text Representations, Authorship Clustering, Silhouette, Regression, Beta distribution, Fusions, Z-Score}
\end{abstract}
\setcounter{page}{5}
