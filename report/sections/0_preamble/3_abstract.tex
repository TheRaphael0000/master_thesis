\begin{abstract}
  \thispagestyle{plain}
  \setcounter{page}{4}
  This study aim to select different methods, text representations and distance metrics to implement and evaluate different strategies to solve the authorship clustering problem.
  In this case, having a set of $n$ texts written by an unknown number of authors, the task is to regroup under the same cluster every texts written by the same author.
  The number of different author denoted $k$ can variate from 1 (all texts are written by the same author) to $n$ (each text is written by a distinct author).

  To represent the underlying style of each document various text representation have been suggested: words frequencies, lemmas frequencies, letters $n$-grams frequencies or POS sequences frequencies.
  Each text is then represented by a vector with a number of dimension corresponding to the number of features.
  $L^1$, $L^2$ and inner product based distances are used in conjunction to compute distance between vectors representations.
  In addition to the vector representation, compression techniques are also applied.
  The Oxquarry, Brunet and St-Jean corpora are used to evaluate the system.

  In a first part, the rank lists obtained by each individual representation are used to solve the authorship clustering task.
  From a rank list, an automatic clustering algorithm can generate the corresponding clusters, hopefully regrouping all the texts written by a given author under the same cluster.
  The clustering task is achieved by using three different models based on hierarchical clustering.
  A Silhouette score based model, a distribution-based model and a regression-based clustering models are proposed and evaluated.
  The clustering obtained by these models are close to the best achievable clustering for the rank lists used.
  In a second part, a new rank list is created by fusing rank lists obtained using different strategies.
  Two approaches are explored for the fusion: one use Z-Score normalization and another one use logistic regression.
  The rank list obtained by fusion is shown to have better performances than the best individual rank list.
  In addition to the fusion, two veto strategy are proposed to try to enhance the fusion quality.
  The veto do not show any significant improvements with rank lists used.
  Lastly, we showed that using rank lists obtained by fusion yield better clustering results than the ones obtained by individual methods.

  {\small \textbf{\textit{Keywords---}} Rank List, Text Representations, Authorship Clustering, Silhouette, Regression, Beta distribution, Fusions, Z-Score}
\end{abstract}
\setcounter{page}{5}
