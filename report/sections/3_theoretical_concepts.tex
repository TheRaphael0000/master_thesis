\section{Theoretical concepts}

This section contain basic notions commonly used in the authorship attribution and author profiling literature.
The definitions in this sections are an adapted version of the ones in~\cite{savoy_stylo},~\cite{kocher_linking},~\cite{bcubed}.
Some others are specifically related to this study.


\begin{definition}[Document]
  A document $X_i$ is a ordered list of token. A token is a non empty string. Example:
  \begin{equation}
    X_i = ("\text{the}", "\text{quick}", "\text{brown}", "\text{fox}", "\text{.}")
  \end{equation}
\end{definition}

\begin{definition}[Author]
  An author $Y_i$ is a string describing the author. Example:
  \begin{equation}
    Y_i = "\text{Zola}"
  \end{equation}
\end{definition}

\begin{definition}[Corpus]
  The corpus is a list of document X of size s and a list of authors Y of size k.
  \begin{equation}
    X = (X_1, X_2, X_3, X_{...}, X_s)
  \end{equation}
  \begin{equation}
    N = |X|
  \end{equation}
  \begin{equation}
    Y = (Y_1, Y_2, Y_3, Y_{...}, Y_k)
  \end{equation}
  \begin{equation}
    k = |Y|
  \end{equation}
\end{definition}

\begin{definition}[Text Authorship]
  The function $f$, is a surjective-only function which map the every text $X$ to a single author $Y$
  \begin{equation}
    Y = f(X)
  \end{equation}
  The set of $Y_a$ is the set of document written by $a$.
  \begin{equation}
    \hat{Y}_a = \{X_i | f(X_i) = a\}
  \end{equation}
  \begin{equation}
    N = \sum_{i} |\hat{Y}_i|
  \end{equation}
\end{definition}

\begin{definition}[Relevant set]
  The relevant set contain every possible pairs (also called links) of document with the same authors. Links in this set are called "true links" in this study.
  \begin{equation}
    R = \{(X_a, X_b) | f(X_a) = f(X_b) \land X_a \neq X_b) \forall (X_a, X_b)\}
  \end{equation}
\end{definition}

\begin{definition}[Ranked list]
  A ranked list is a ordered list of pair of document.
  \begin{equation}
    L = ((X_a, X_b) | X_a \neq X_b \forall (X_a, X_b))
  \end{equation}
  \begin{equation}
    |L| = \frac{N \cdot (N - 1)}{2}
  \end{equation}
\end{definition}

\begin{definition}[$N$-Grams]
  A $N$-gram is a special type of tokenization which is constructed by creating a token for the substrings from $0$ to \textit{text\_size} - $N$ of size $N$.
  Example: Using 3-grams The string: "brown fox" is converted to: \\
  $("\text{bro}", "\text{row}", "\text{own}", "\text{wn\_}", "\text{n\_f}", "\text{\_fo}", "\text{fox}")$
\end{definition}

\begin{definition}[r ratio~\cite{pan16}]
  The ratio between the number of clusters k and the number of documents N in a given corpus.
  If r is close to 0, most documents are in multi-documents clusters and there is a great density of relevant links.
  In this other hand if r is close to 1, most of the document belong to single document clusters and there are few relevant links.
  \begin{equation}
    r = \frac{k}{N}
  \end{equation}
  The inverse of the r ratio is equaivalent to the mean number of documents per authors.
  \begin{equation}
    \frac{1}{r} = \frac{N}{k} = \frac{1}{k} \cdot \sum_{i} |\hat{Y}_i|
  \end{equation}
\end{definition}

\subsection{Document distances}

\begin{definition}[Manhanttan Distance]
  To compute the manhanttan distance the following equation is used.
  \begin{equation}
    dist_{Manhanttan}(A, B) = \sum_{i=1}^{m} |a_i - b_i|
  \end{equation}
\end{definition}

\subsection{Rank list evaluation}

\begin{definition}[Relevant link~\cite{kocher_linking}]
  A relevant link is a link in the relevant set.
  \begin{equation}
    relevant(l_i) =
    \begin{cases}
      1, & if\ l_i \in R \\
      0, & otherwise
    \end{cases}
  \end{equation}
\end{definition}

\begin{definition}[Precision@k~\cite{kocher_linking}]
  The precision@k is a function which take a integer k, with k < |L|
  \begin{equation}
    precision(k) = \frac{1}{k} \sum_{j=1}^{k} relevant(j)
  \end{equation}
\end{definition}

\begin{definition}[Average Precision (AP)]
  The mean over the precision@k each time a relevant link is retrieved.
  \begin{equation}
    AveragePrecision = \frac{1}{|R|} \sum_{j=1}^{|L|} precision(j) \cdot relevant(j)
  \end{equation}
\end{definition}

\begin{definition}[RPrec~\cite{kocher_linking}]
  The RPrec is the precision in the rank list at rank |R|.
  With R being the relevant set. (Not to be confused with the r ratio)
  \begin{equation}
    RPrec = precision(|R|)
  \end{equation}
\end{definition}

\begin{definition}[HPrec~\cite{kocher_linking}]
  The HPrec represent a maximal rank j in the rank list where the precision is still 100\%.
  This value is in the range [0 - |R|].
  0 means the first pair in the rank list is incorrect.
  |R| means every true links are ranked in the top part of the rank list.
  \begin{equation}
    HPrec = \max\{i \in \mathbf{N} | precision(i) = 1\}
  \end{equation}
\end{definition}

\subsection{Clustering evaluation}

\begin{definition}[Correctness~\cite{bcubed}]
  Let L(e) and C(e) be the category and the cluster of an element e.
  The correctness has a value of one if the two elements are in the both in the same cluster and has the same category OR both in a different cluster and a different category. ($A \Longleftrightarrow B \equiv (A \land B) \lor (\neg A \land \neg B)$)
  \begin{gather*}
    Correctness(e, e') = \\
    \begin{cases}
      1, & if (L(e) = L(e')) \Longleftrightarrow (C(e) = C(e'))\\
      0, & otherwise
    \end{cases}
  \end{gather*}
\end{definition}

\begin{definition}[Precision $BCubed$~\cite{bcubed}]
  The average of correctness for all elements on the average of all element such that their cluster is the same.
  \begin{equation}
    BCubed_{precision} = \text{Avg}_{e}[\text{Avg}_{e' C(e)=C(e')}[Correctness(e, e')]]
  \end{equation}
\end{definition}

\begin{definition}[Recall $BCubed$~\cite{bcubed}]
  The average of correctness for all elements on the average of all element such that their category is the same.
  \begin{equation}
    BCubed_{recall} = \text{Avg}_{e}[\text{Avg}_{e' L(e)=L(e')}[Correctness(e, e')]]
  \end{equation}
\end{definition}

\begin{definition}[$BCubed F_1$ Score~\cite{bcubed}]
  The harmonic mean between the $BCubed_{precision}$ and $BCubed_{recall}$
  \begin{equation}
    BCubed_{F_1} =
    2 \cdot \frac{BCubed_{precision} \cdot BCubed_{recall}}
    {BCubed_{precision} + BCubed_{recall}}
  \end{equation}
\end{definition}
