\section{Theoretical concepts}

This section contain basic notions commonly used in the authorship attribution and author profiling literature.

\begin{definition}[Document]
  A document $X$ is a ordered list of token. A token is a non empty string. Example:
  \begin{equation}
    X = ("\text{the}", "\text{quick}", "\text{brown}", "\text{fox}", "\text{.}")
  \end{equation}
\end{definition}

\begin{definition}[Author]
  An author $Y$ is a string describing the author. Example:
  \begin{equation}
    Y = "\text{Zola}"
  \end{equation}
\end{definition}

\begin{definition}[Corpus]
  The corpus is a set of document containing $S$ documents.
  \begin{equation}
    C = {X_1, X_2, X_3, X_{...}, X_S}
  \end{equation}
\end{definition}

\begin{definition}[$N$-Grams]
  A $N$-gram is a special type of tokenization which is constructed by creating a token for the substrings from $0$ to \textit{text\_size} - $N$ of size $N$.
  Example: Using 3-grams The string: "brown fox" is converted to: \\
  $("\text{bro}", "\text{row}", "\text{own}", "\text{wn\_}", "\text{n\_f}", "\text{\_fo}", "\text{fox}")$
\end{definition}

\begin{definition}[Text Authorship]
  The function $f$, is a surjective-only function which map the every text $X$ to a single author $Y$
  \begin{equation}
    Y = f(X)
  \end{equation}
  The set of $Y_a$ is the set of document written by $a$.
  \begin{equation}
    Y_a = \{X_i | f(X_i) = a\}
  \end{equation}
\end{definition}

\begin{definition}[Relevant set]
  The relevant set contain every possible pairs (also called links) of document with the same authors.
  \begin{equation}
    R = \{(X_a, X_b) | f(X_a) = f(X_b) \land X_a \neq X_b) \forall (X_a, X_b)\}
  \end{equation}
\end{definition}

\begin{definition}[Ranked list]
  A ranked list is a ordered list of pair of document.
  \begin{equation}
    L = ((X_a, X_b) | X_a \neq X_b \forall (X_a, X_b))
  \end{equation}
  \begin{equation}
    |L| = \frac{|C| \cdot (|C| - 1)}{2}
  \end{equation}
\end{definition}

\subsection{Documentpp comparaison}

\begin{definition}[Manhanttan Distance]
  To compute the manhanttan distance the following equation is used.
  \begin{equation}
    dist_{Manhanttan}(A, B) = \sum_{i=1}^{m} |a_i - b_i|
  \end{equation}
\end{definition}

\subsection{Rank list evaluation}

\begin{definition}[Relevant link]
  A relevant link is a link in the relevant set.
  \begin{equation}
    relevant(l_i) =
    \begin{cases}
      1, & if\ l_i \in R \\
      0, & otherwise
    \end{cases}
  \end{equation}
\end{definition}

\begin{definition}[Precision@k]
  The precision@k is a function which take a integer k, with k < |L|
  \begin{equation}
    precision(k) = \frac{1}{k} \sum_{j=1}^{k} relevant(j)
  \end{equation}
\end{definition}

\begin{definition}[Average Precision (AP)]

  \begin{equation}
    AP = \frac{1}{|L|} \sum_{j=1}^{|L|} precision(j) \cdot relevant(j)
  \end{equation}
\end{definition}

\begin{definition}[RPrec]
  The RPrec is the precision in the rank list at rank |R|.
  With R being the relevant set.
  \begin{equation}
    RPrec = precision(|R|)
  \end{equation}
\end{definition}

\begin{definition}[HPrec]
  The HPrec represent a maximal rank j in the rank list where the precision is still 100\%.
  This value is in the range [0 - |R|].
  0 means the first pair in the rank list is incorrect.
  |R| means every true links are ranked in the top part of the rank list.
  \begin{equation}
    HPrec = \max\{i \in \mathbf{N} | precision(i) = 1\}
  \end{equation}
\end{definition}

\subsection{Clustering evaluation}

\begin{definition}[Precision $BCubed$]

  \begin{equation}
  \end{equation}
\end{definition}

\begin{definition}[Recall $BCubed$]

  \begin{equation}
  \end{equation}
\end{definition}

\begin{definition}[$BCubed F_1$ Score]

  \begin{equation}
  \end{equation}
\end{definition}
