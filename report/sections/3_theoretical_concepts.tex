\section{Theoretical concepts}

This section contain basic notions commonly used in the authorship attribution, authorship verification and authorship clustering.

\begin{definition}[Document]
  A document $X_i$ is a ordered list of token. A token is a non empty string. Example:
  \begin{equation}
    X_i = ("\text{the}", "\text{quick}", "\text{brown}", "\text{fox}", "\text{.}")
  \end{equation}
  To obtain tokens from a long string contraining a non-tokenized document, a tokenizer i needed.
\end{definition}

\begin{definition}[Author]
  An author $Y_i$ is a string describing the author. Example:
  \begin{equation}
    Y_i = "\text{Zola}"
  \end{equation}
\end{definition}

\begin{definition}[Corpus]
  The corpus is a list of document X of size N and a list of authors Y of size k.
  \begin{equation}
    X = (X_1, X_2, X_3, X_{...}, X_N)
  \end{equation}
  \begin{equation}
    N = |X|
  \end{equation}
  \begin{equation}
    Y = (Y_1, Y_2, Y_3, Y_{...}, Y_k)
  \end{equation}
  \begin{equation}
    k = |Y|
  \end{equation}
\end{definition}

\begin{definition}[Text Authorship]
  The function $f$, is a surjective-only function which map the every text $X$ to a single author $Y$
  \begin{equation}
    Y = f(X)
  \end{equation}
  The set of $\hat{Y}_a$ is the set of document written by $a$.
  \begin{equation}
    \hat{Y}_a = \{X_i | f(X_i) = a\}
  \end{equation}
  \begin{equation}
    N = \sum_{i} |\hat{Y}_i|
  \end{equation}

\end{definition}

\begin{definition}[Relevant set]
  The relevant set contain every possible pairs (also called links) of document with the same authors. Links in this set are called \textit{true links} in this study. Every other pairs are called \textit{false links}, since they don't have the same authors.
  \begin{equation}
    R = \{(X_a, X_b) | f(X_a) = f(X_b) \land X_a \neq X_b) \forall (X_a, X_b)\}
  \end{equation}
\end{definition}

\begin{definition}[r ratio~\cite{pan16}]
  The ratio between the number of clusters k and the number of documents N in a given corpus.
  \begin{equation}
    r = \frac{k}{N}
  \end{equation}
  The inverse of the r ratio is equaivalent to the mean number of documents per authors.
  \begin{equation}
    \frac{1}{r} = \frac{N}{k} = \frac{1}{k} \cdot \sum_{i} |\hat{Y}_i|
  \end{equation}
  If r is close to 0, most documents are in multi-documents clusters and there is a great density of true links.
  In this other hand, if r is close to 1, most of the document belong to single document clusters and there are few true links.
\end{definition}

\begin{definition}[true links' ratio]
  The ratio between the number of true links |R| and the number of links |L| in a given corpus.
  \begin{equation}
    tl_r = \frac{|R|}{|L|}
  \end{equation}
\end{definition}
