\subsection{Rank lists}

Rank lists are used to order objects such that the most interesting object is at the top and every subsequent object become less interesting.
In information retrieval systems, rank lists are used to order the results from the most relevant to the user's query to the least relevant.
For the authorship verification problem, the rank list can also be used, see Definition~\ref{def:rank_list}.

\begin{definition}[Ranked list for authorship verification \label{def:rank_list}]
  \begin{equation}
    L = (\left[(X_a, X_b): Score(X_a, X_b)\right] | X_a \neq X_b \forall (X_a, X_b))
  \end{equation}
  \begin{equation}
    |L| = \frac{N \cdot (N - 1)}{2}
  \end{equation}
\end{definition}

A ranked list for the authorship verification problem is an ordered list containing document pairs and a score for the pair.
In most cases, the rank list contain every possible pairs of documents.

The ranked list is ordered by the score, such that the most similar documents are at the top of the list.
When the scoring function is based on a distance metrics, the increasing order is used.
For the scoring function based on similarity, the decreasing order is used.
The most similar documents are the most likely they are written by the same author.
Thus, the top ranks should contain pairs of documents written by the same author.

The computational cost for evaluating a rank list is $\frac{n \cdot (n - 1)}{2}$ multiplied by the computational cost of the score function.
Since the complexity of the score function is lower than $O(n^2)$.
The space complexity is also $O(n^2)$.

Example~\ref{ex:rank_list} show a simple example using two-dimensional vectors and the Manhattan distance (presented in Section~\ref{sec:fv_distances}) to create a rank list.

\begin{example}
  \centering
  \caption{Rank list computation using two-dimensional vectors and the Manhattan distance}
  \label{ex:rank_list}

  \begin{subexample}{\linewidth}
    \centering
    \subcaption{List of two-dimensional vectors}
    \begin{tabular}{l r r}
      \toprule
      Vector ID & Vector \\
      \midrule
      0 & $[0, 0]$ \\
      1 & $[1, 2]$ \\
      2 & $[4, 6]$ \\
      3 & $[1, 4]$ \\
      \bottomrule
    \end{tabular}
  \end{subexample}

  \vspace{0.5cm}

  \begin{subexample}{\linewidth}
    \centering
    \subcaption{Pairwise manhanttan distances}
    \begin{tabular}{l l}
      \toprule
      Vector Pair IDs & $dist_{Manhattan}(A, B)$ \\
      \midrule
      (0, 1) & $|0-1| + |0-2| = 3$ \\
      (0, 2) & $|0-4| + |0-6| = 10$ \\
      (0, 3) & $|0-1| + |0-4| = 5$ \\
      (1, 2) & $|1-4| + |2-6| = 7$ \\
      (1, 3) & $|1-1| + |2-4| = 2$ \\
      (2, 3) & $|4-1| + |6-4| = 5$ \\
      \bottomrule
    \end{tabular}
  \end{subexample}

  \vspace{0.5cm}

  \begin{subexample}{\linewidth}
    \centering
    \subcaption{Ordered rank list by distances}
    \begin{tabular}{l c r}
      \toprule
      Rank & Vector Pair IDs & $dist_{Manhattan}(A, B)$ \\
      \midrule
      1st   & (1, 3) & $2$ \\
      2nd   & (0, 1) & $3$ \\
      3-4rd & (2, 3) & $5$ \\
      3-4th & (0, 3) & $5$ \\
      5th   & (1, 2) & $7$ \\
      6th   & (0, 2) & $10$ \\
      \bottomrule
    \end{tabular}
  \end{subexample}

\end{example}
