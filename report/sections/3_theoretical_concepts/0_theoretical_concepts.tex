\section{Theoretical concepts \label{sec:theoretical_concepts}}

This section contain basic notions commonly used in the authorship attribution, authorship verification and authorship clustering.

\begin{definition}[Document]
  A document $X_i$ is a ordered list of token. A token is a non empty string. Example:
  \begin{equation}
    X_i = ("\text{the}", "\text{quick}", "\text{brown}", "\text{fox}", "\text{.}")
  \end{equation}
  To obtain tokens from a long string contraining a non-tokenized document, a tokenizer i needed.
\end{definition}

\begin{definition}[Author]
  An author $Y_j$ is a string describing the author. Example:
  \begin{equation}
    Y_j = "\text{Zola}"
  \end{equation}
  Since this study is related to clustering, authors are also called clusters.
\end{definition}

\begin{definition}[Corpus]
  A corpus contain two lists $X$ and $Y$.
  $X$ contain is a list of documents $X_i$ of size $N$ and $Y$ a list of authors $Y_j$ of size $k$.
  \begin{gather}
    X = (X_1, X_2, X_3, X_{...}, X_N) \\
    N = |X|
  \end{gather}
  \begin{gather}
    Y = (Y_1, Y_2, Y_3, Y_{...}, Y_k) \\
    k = |Y|
  \end{gather}
\end{definition}

\begin{definition}[Text Authorship]
  The function $f$, is a surjective-only function which map every text $X_i$ to a single author $Y_j$
  \begin{equation}
    Y_j = f(X_i)
  \end{equation}
  The set of $\hat{Y}_a$ is the set of document written by $Y_a$.
  \begin{equation}
    \hat{Y}_a = \{X_i | f(X_i) = a\}
  \end{equation}
  \begin{equation}
    N = \sum_{j}^k |\hat{Y}_j|
  \end{equation}

\end{definition}

\begin{definition}[Relevant set]
  The relevant set $R$ contain every possible pairs of document with the same authors.
  Links in this set are called \textit{true links} in this study.
  \begin{equation}
    \label{def:relevant_set}
    R = \{(X_a, X_b)\ |\ \left( f(X_a) = f(X_b) \right) \land \left(X_a \neq X_b \right) \forall (X_a, X_b)\}
  \end{equation}
  The non-relevant set $R$ contain every \textit{false links}.
  Every links are contain in $L$, the union of the relevant and non-relevant set.
  \begin{equation}
    \bar{R} = \{(X_a, X_b)\ |\ \left( f(X_a) \neq f(X_b) \right) \land \left(X_a \neq X_b \right) \forall (X_a, X_b)\}
  \end{equation}
  \begin{gather}
    R \cap \bar{R} = \emptyset \\
    L = \bar{R} \cup R \\
    |L| = \frac{N * (N-1)}{2}
  \end{gather}
\end{definition}

\begin{definition}[r ratio~\cite{pan16}]
  The r ratio is the ratio between the number of different authors $k$ and the number of documents $N$ in a given corpus.
  \begin{equation}
    r = \frac{k}{N}
  \end{equation}
  The inverse of the r ratio is equaivalent to the mean number of documents per authors.
  \begin{equation}
    \frac{1}{r} = \frac{N}{k} = \frac{1}{k} \cdot \sum_{i} |\hat{Y}_j|
  \end{equation}
  If $r$ is close to $0$, most documents are written by different authors and there is a great density of true links.
  In this other hand, if $r$ is close to $1$, most of the document are written by a single authors and there are few true links.
\end{definition}

\begin{definition}[true links' ratio]
  The true links' ratio is the ratio between the number of true links $|R|$ and the number of links $|L|$ in a given corpus.
  \begin{equation}
    tl_r = \frac{|R|}{|L|}
  \end{equation}
\end{definition}
